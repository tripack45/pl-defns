\documentclass[10pt]{article}

\usepackage{etoolbox}
\usepackage{envar}

\title{\texttt{envar} Package Example}
\author{Yue Yao}
\date{\today}

\begin{document}

\maketitle{}

\section*{Overview}

This document demonstrates the usage of the \texttt{envar} package.
Include the declaration \verb|\usepackage{envar}| to define macros for accessing environment variables.

\section*{Usage}

The package provides the \verb|\GetEnv| and \verb|\BindEnv| commands.

\subsection*{\texttt{\textbackslash GetEnv}: Direct Output}

\begin{itemize}
\item You can directly print the value of an environment variable like \texttt{HOME}. 
\verb|\GetEnv{HOME}| will produce: 
\texttt{\GetEnv{HOME}}.

\item You can also provide a default value as an optional argument, in case the 
variable is not set, or set to empty.

\verb|\GetEnv[default]{NON_EXISTENT_VARIABLE}| produces: 
\texttt{\GetEnv[default]{NON_EXISTENT_VARIABLE}}.

\item If the environment variable is not set, and the default is omitted, it produces 
a default value \texttt{\GetEnv{NON_EXISTENT_VARIABLE}}. Testing this value is tricky, however.
Therefore, should the need arises, the user is recommended to use the \verb|\BindEnv| variant.
\end{itemize}

\subsection*{\texttt{\textbackslash BindEnv}: Storing the value}

You can store the value of an environment variable in a macro for later use 
with \verb|\BindEnv|. Here we store the value of \texttt{TEXMFDIST} in a new command 
called \verb|\texmfdistpath|:
%
\begin{quotation}
    \verb|\BindEnv{\texmfdistpath}{TEXMFDIST}|
\end{quotation}

\BindEnv{\texmfdistpath}{TEXMFDIST}

Now we can reuse it: the value is \texttt{\texmfdistpath}.

\begin{itemize}
\item As with \verb|\GetEnv|, the command takes an optional argument of default value, in case 
the environment variable is nonexistent:

\verb|\BindEnv{\nonexistent}[default]{NON_EXISTENT_VARIABLE}|

\BindEnv{\nonexistent}[default]{NON_EXISTENT_VARIABLE}

Accessing \verb|\nonexistent| gives \texttt{\nonexistent}.

\item If the environment variable is empty or not set, and the default argument is omitted

\verb|\BindEnv{\another}{NON_EXISTENT_VARIABLE2}|

\BindEnv{\another}{NON_EXISTENT_VARIABLE2}

Then \verb|\another| is bound to the empty token list, which can be tested 
with \verb|\ifdefempty| from \texttt{etoolbox} as follows:
%
\begin{verbatim}
\ifdefempty{\another}{Argument is empty.}{Argument is not empty.}
\end{verbatim}
%
Result: \texttt{\ifdefempty{\another}{Argument is empty.}{Argument is not empty.}}
\end{itemize}

\subsection*{Conveniences}

The package offers a few conveniences:

\begin{itemize}
\item You can quickly check if an envar is defined (holds a non-empty value) by
%
\begin{verbatim}
\EnvHasValueTF{HOME}{HOME has value}{impossible}
\EnvHasValueTF{HOMER}{unlikely}{HOMER does not have value}
\end{verbatim}
%
Results:
\texttt{\EnvHasValueTF{HOME}{ HOME has value }{ impossible }} and
\texttt{\EnvHasValueTF{HOMER}{ unlikely }{ HOMER does not have value }}.

\item If you want to check if an envar is defined, and in the case that it is defined 
work with the value, use \verb|\BindEnvHasValueTF|:

\begin{verbatim}
\BindEnvHasValueTF{\home}{HOME}
{ Home has value: \home }
{ You have no HOME? }
\end{verbatim}

Results: \texttt{%
\BindEnvHasValueTF{\home}{HOME}
{ Home has value: \home }
{ You have no HOME? }
}

If the envar holds no value, then the macro holds an empty token list.

\begin{verbatim}
\BindEnvHasValueTF{\homer}{HOMER}
{ Homer has value: \homer }
{ \ifdefempty{\homer}{Empty}{NotEmpty} }
\end{verbatim}

Results: \texttt{%
\BindEnvHasValueTF{\homer}{HOMER}
{ Homer has value: \homer }
{ \ifdefempty{\homer}{Empty}{NotEmpty} }
}

\end{itemize}

\end{document}